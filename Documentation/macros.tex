\usepackage{ifthen}
\usepackage{tikz}
\usepackage{xspace}
%\usepackage{minted}
\usepackage{listings}
\usepackage[T1]{fontenc}






%Define color environments:

\newenvironment{DK}[1]{{\color{gray}Commend from D: #1}}

\newenvironment{DKnew}[1]{{\color{blue}NEW from D: #1}}

\newenvironment{DKres}[1]{{\color{BrickRed}RESTRUCTURED from D: #1}}


\definecolor{mediumjunglegreen}{rgb}{0.11, 0.21, 0.18}
\definecolor{bg}{HTML}{282828}

\lstset{
	escapeinside={?}{?},
	basicstyle=\small\ttfamily,%
	columns=fullflexible,%
	keywordstyle={\color{blue!80!black}},
	keywordstyle=[3]{\color{yellow!50!black}},%
	numbers=left,                    % where to put the line-numbers; possible values are (none, left, right)
	numbersep=5pt,                   % how far the line-numbers are from the code
	numberstyle=\tiny, % the style that is used for the line-number
	commentstyle=\fontfamily{palatino}\color{purple!40!black},
	breaklines=false,
	frame=tb,
	tabsize=3
}
%

%add more keywords and styles
\lstset{
	emphstyle={\color{blue!80!black}}, emph={include,std,string,constexpr,array,vector,True,False},
	emph=[2]{mimes,nsc,util,Evolution,Cosmo,Timer,nscSove}, emphstyle=[2]{\color{red!80!black}},
	emph=[3]{Solver,Method,LD,METHOD,SOLVER}, emphstyle=[3]{\color{green!80!black}}
}
	
	
\newcommand*\cppin{\lstinline[language=c++]}

\newcommand*\pyin{\lstinline[language=c++]}

\lstnewenvironment{cpp}
{\lstset{language=c++}}
{}

\lstnewenvironment{py}
{\lstset{language=Python}}
{}

\lstnewenvironment{bash}
{\lstset{language=Python}}
{}

\lstnewenvironment{pseudo}
{\lstset{language=Python}}
{}


%\newenvironment{cpp}{\usemintedstyle{xcode}\VerbatimEnvironment\begin{minted}[frame=lines,framesep=5mm,baselinestretch=1.2,fontsize=\footnotesize,linenos,mathescape,escapeinside=??]{cpp}}{\end{minted}} 

%\newenvironment{py}{\usemintedstyle{xcode}\VerbatimEnvironment\begin{minted}[frame=lines,framesep=5mm,baselinestretch=1.2,fontsize=\footnotesize,linenos,mathescape,escapeinside=??]{py}}{\end{minted}} 

%\newenvironment{bash}{\usemintedstyle{xcode}\VerbatimEnvironment\begin{minted}[frame=lines,framesep=5mm,baselinestretch=1.2,fontsize=\footnotesize,linenos,mathescape]{bash}}{\end{minted}} 

%in order to writ maths outside of comments put them inside ?$$?. Theta is for x^2, write ?$x^2$?.
%\newenvironment{pseudo}{\usemintedstyle{xcode}\VerbatimEnvironment\begin{minted}[frame=lines,framesep=5mm,baselinestretch=1.2,fontsize=\footnotesize,linenos,escapeinside=??,mathescape=true]{python}}{\end{minted}} 

\newenvironment{run}[1]{``{\tt #1}"\xspace}


\newcommand{\dint}{  \displaystyle \int }
%%%%%%%%%%%%%%%%%%%%%%%%%%%%%%%%%%%%%%%%%%
\newcommand{\ie}{{\em i.e.}\xspace}
\newcommand{\eg}{{\em e.g.}\xspace}
\newcommand{\GeV}{{\rm GeV}\xspace}
\newcommand{\TeV}{{\rm TeV}\xspace}
\newcommand{\MeV}{{\rm MeV}\xspace}
\newcommand{\keV}{{\rm keV}\xspace}

\def\nsc{{\tt NSC++}\xspace}
\def\mimes{{\tt MiMeS}\xspace}

\newcommand{\CPP}{{\tt C++}\xspace}

\newcommand{\PY}{{\tt python}\xspace}

\newcommand{\JUPY}{{\tt jupyter}\xspace}

\newcommand{\rhs}{RHS\xspace}
\newcommand{\lhs}{LHS\xspace}

\newcommand{\sR}{ s_{\rm R}\xspace}
\newcommand{\HR}{ H_{\rm R}\xspace}
\newcommand{\Hend}{ H_{\rm R,\,end}\xspace}
\newcommand{\rhoR}{ \rho_{\rm R}\xspace}
\newcommand{\rhoPhi}{ \rho_{\Phi}\xspace}
\newcommand{\GammaPhi}{ \Gamma_{\Phi}\xspace}


\newcommand{\fR}{ f_{\rm R}\xspace}
\newcommand{\fPhi}{f_{\Phi}\xspace}

\newcommand{\sRi}{ s_{\rm R\,i}\xspace}
\newcommand{\rhoRi}{ \rho_{\rm R\,i}\xspace}
\newcommand{\rhoPhii}{ \rho_{\Phi\,{\rm i}}\xspace}



\newcommand{\geff}{ g_{\rm eff}{}\xspace}
\newcommand{\heff}{ h_{\rm eff}{}\xspace}
\renewcommand{\dh}{ \delta_h{}\xspace}

\newcommand{\Ham}{ \mathcal{H}\xspace}


\newcommand{\Ti}{ T_{\rm i}{}\xspace}

\newcommand{\ti}{ t_{\rm i}{}\xspace}

\newcommand{\ui}{ u_{\rm i}{}\xspace}

\newcommand{\ai}{ a_{\rm i}{}\xspace}

\newcommand{\Tend}{ T_{\rm end}{}\xspace}

\newcommand{\tend}{ t_{\rm end}{}\xspace}

\newcommand{\uend}{ u_{\rm end}{}\xspace}

\newcommand{\aend}{ a_{\rm end}{}\xspace}

\newcommand{\TEI}{ T_{\rm E1}{}\xspace}
\newcommand{\TEII}{ T_{\rm E2}{}\xspace}
\newcommand{\TDI}{ T_{\rm D1}{}\xspace}
\newcommand{\TDII}{ T_{\rm D2}{}\xspace}

\newcommand{\uEI}{ u_{\rm E1}{}\xspace}
\newcommand{\uEII}{ u_{\rm E2}{}\xspace}
\newcommand{\uDI}{ u_{\rm D1}{}\xspace}
\newcommand{\uDII}{ u_{\rm D2}{}\xspace}

\newcommand{\EI}{{\rm E1}{}\xspace}
\newcommand{\EII}{{\rm E2}{}\xspace}
\newcommand{\DI}{{\rm D1}{}\xspace}
\newcommand{\DII}{{\rm D2}{}\xspace}


\newcommand{\vev}[1]{\langle #1 \rangle}
\newcommand{\Bvev}[1]{\Bigg\langle #1 \Bigg\rangle}
\newcommand{\bvev}[1]{\Big\langle #1 \Big\rangle}




\newcommand{\lrb}[1]{\left( #1 \right)}
\newcommand{\lrsb}[1]{\left[ #1 \right]}
\newcommand{\lrBigb}[1]{\Big( #1 \Big)}
\newcommand{\lrBigsb}[1]{\Big[ #1 \Big]}
\newcommand{\lrBiggb}[1]{\Bigg( #1 \Bigg)}
\newcommand{\lrBiggsb}[1]{\Bigg[ #1 \Bigg]}

\newcommand{\lrBigcb}[1]{\Big\{ #1 \Big\}}
\newcommand{\lrBiggcb}[1]{\Bigg\{ #1 \Bigg\}}
%%%%%%%%%%%%%%%%%%%%%%%%%%%%%%%%%%%%%%%%%

%%%%%%%%%%%%%%%%%%%%%%%%%%%%%%%%%%%%%%%%%%%%%%%%%%%--Begin_refs--%%%%%%%%%%%%%%%%%%%%%%%%%%%%%%%%%%%%%%%%%%%%%%%%%%%%%%%%%%%%%%%%%%%%%%
\newcounter{NumArgs}

%Define reference to an arbitrary number of equations (\eqs{label_1,label_2....,label_n} will show eqs. ref_1, ref_2, ..., and ref_n)
\newcommand{\eqs}[1]{\setcounter{NumArgs}{0}\foreach\i in{#1}{\stepcounter{NumArgs}}%
	\ifthenelse{\equal{\theNumArgs}{1}}{eq.~(\ref{#1})}%
	{\ifthenelse{\equal{\theNumArgs}{2}}%
		{eqs.~\foreach\i[count=\q]in{#1}{\ifthenelse{\equal{\q}{\theNumArgs}}{and (\ref{\i})}{(\ref{\i})~}}}%
		{eqs.~\foreach\i[count=\q]in{#1}{\ifthenelse{\equal{\q}{\theNumArgs}}{and (\ref{\i})}{(\ref{\i}),~}}}}}


%Define reference to an arbitrary number of equations (\Eqs{label_1,label_2....,label_n} will show Eqs. ref_1, ref_2, ..., and ref_n)
\newcommand{\Eqs}[1]{\setcounter{NumArgs}{0}\foreach\i in{#1}{\stepcounter{NumArgs}}%
	\ifthenelse{\equal{\theNumArgs}{1}}{Eq.~(\ref{#1})}%
	{\ifthenelse{\equal{\theNumArgs}{2}}%
		{Eqs.~\foreach\i[count=\q]in{#1}{\ifthenelse{\equal{\q}{\theNumArgs}}{and (\ref{\i})}{(\ref{\i})~}}}%
		{Eqs.~\foreach\i[count=\q]in{#1}{\ifthenelse{\equal{\q}{\theNumArgs}}{and (\ref{\i})}{(\ref{\i}),~}}}}}


%Define reference to an arbitrary number of labels (\REF{label_1,label_2....,label_n} will show ref_1, ref_2, ..., and ref_n)
\newcommand{\refs}[1]{\setcounter{NumArgs}{0}\foreach\i in{#1}{\stepcounter{NumArgs}}%
	\ifthenelse{\equal{\theNumArgs}{1}}{(\ref{#1})}%
	{\ifthenelse{\equal{\theNumArgs}{2}}%
		{\foreach\i[count=\q]in{#1}{\ifthenelse{\equal{\q}{\theNumArgs}}{and (\ref{\i})}{(\ref{\i})~}}}%
		{\foreach\i[count=\q]in{#1}{\ifthenelse{\equal{\q}{\theNumArgs}}{and (\ref{\i})}{(\ref{\i}),~}}}}}



%Define reference to an arbitrary number of figs (\Figs{label_1,label_2....,label_n} will show ref_1, ref_2, ..., and ref_n)
\newcommand{\Figs}[1]{\setcounter{NumArgs}{0}\foreach\i in{#1}{\stepcounter{NumArgs}}%
	\ifthenelse{\equal{\theNumArgs}{1}}{Figure~(\ref{#1})}%
	{\ifthenelse{\equal{\theNumArgs}{2}}%
		{Figures~\foreach\i[count=\q]in{#1}{\ifthenelse{\equal{\q}{\theNumArgs}}{and (\ref{\i})}{(\ref{\i})~}}}%
		{Figures~\foreach\i[count=\q]in{#1}{\ifthenelse{\equal{\q}{\theNumArgs}}{and (\ref{\i})}{(\ref{\i}),~}}}}}




%Define reference to an arbitrary number of "general reference" (\Gen{message}{label_1,label_2....,label_n} will show message.(ref_1), (ref_2), ..., and (ref_n)
\newcommand{\Gen}[2]{\setcounter{NumArgs}{0}\foreach\i in{#2}{\stepcounter{NumArgs}}%
	\ifthenelse{\equal{\theNumArgs}{1}}{#1.~(\ref{#2})}%
	{\ifthenelse{\equal{\theNumArgs}{2}}%
		{#1.~\foreach\i[count=\q]in{#2}{\ifthenelse{\equal{\q}{\theNumArgs}}{and (\ref{\i})}{(\ref{\i})~}}}%
		{#1.~\foreach\i[count=\q]in{#2}{\ifthenelse{\equal{\q}{\theNumArgs}}{and (\ref{\i})}{(\ref{\i}),~}}}}}


%%%%%%%%%%%%%%%%%%%%%%%%%%%%%%%%%%%%%%%%%%%%%%%%%%%--End_refs--%%%%%%%%%%%%%%%%%%%%%%%%%%%%%%%%%%%%%%%%%%%%%%%%%%%%%%%%%%%%%%%%%%%%%%