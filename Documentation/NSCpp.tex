\documentclass[11pt,a4paper]{article}
\usepackage[utf8]{inputenc}
\usepackage{amsmath}
\usepackage{amsfonts}
\usepackage{amssymb}
\usepackage[left=2cm,right=2cm,top=2.5cm,bottom=2.5cm]{geometry}


\usepackage{slashed}
\usepackage{hyperref}
\usepackage{graphicx}
\usepackage{caption}
\usepackage{float}

\usepackage{cite}


\usepackage[labelformat=simple]{subcaption}
\renewcommand\thesubfigure{(\alph{subfigure})}
\captionsetup[figure]{format=plain, font={sl}}

\usepackage{multirow}
\usepackage[dvipsnames]{xcolor}
\usepackage{multicol}
\usepackage{authblk}

\renewcommand{\theequation}{\arabic{section}.\arabic{equation}}



\usepackage{ifthen}
\usepackage{tikz}
\usepackage{xspace}
%\usepackage{minted}
\usepackage{listings}
\usepackage[T1]{fontenc}






%Define color environments:

\newenvironment{DK}[1]{{\color{gray}Commend from D: #1}}

\newenvironment{DKnew}[1]{{\color{blue}NEW from D: #1}}

\newenvironment{DKres}[1]{{\color{BrickRed}RESTRUCTURED from D: #1}}


\definecolor{mediumjunglegreen}{rgb}{0.11, 0.21, 0.18}
\definecolor{bg}{HTML}{282828}

\lstset{
	escapeinside={?}{?},
	basicstyle=\small\ttfamily,%
	columns=fullflexible,%
	keywordstyle={\color{blue!80!black}},
	keywordstyle=[3]{\color{yellow!50!black}},%
	numbers=left,                    % where to put the line-numbers; possible values are (none, left, right)
	numbersep=5pt,                   % how far the line-numbers are from the code
	numberstyle=\tiny, % the style that is used for the line-number
	commentstyle=\fontfamily{palatino}\color{purple!40!black},
	breaklines=false,
	frame=tb,
	tabsize=3
}
%

%add more keywords and styles
\lstset{
	emphstyle={\color{blue!80!black}}, emph={include, std, ,string,constexpr,array,vector},
	emph=[2]{mimes,util,Axion,AxionEOM,Cosmo,AxionMass,AnharmonicFactor,Scan,ScanObs,Timer}, emphstyle=[2]{\color{red!80!black}},
	emph=[3]{Solver,Method,LD,METHOD,SOLVER}, emphstyle=[3]{\color{green!80!black}}
}
	
	
\newcommand*\cppin{\lstinline[language=c++]}

\newcommand*\pyin{\lstinline[language=c++]}

\lstnewenvironment{cpp}
{\lstset{language=c++}}
{}

\lstnewenvironment{py}
{\lstset{language=Python}}
{}

\lstnewenvironment{bash}
{\lstset{language=Python}}
{}

\lstnewenvironment{pseudo}
{\lstset{language=Python}}
{}


%\newenvironment{cpp}{\usemintedstyle{xcode}\VerbatimEnvironment\begin{minted}[frame=lines,framesep=5mm,baselinestretch=1.2,fontsize=\footnotesize,linenos,mathescape,escapeinside=??]{cpp}}{\end{minted}} 

%\newenvironment{py}{\usemintedstyle{xcode}\VerbatimEnvironment\begin{minted}[frame=lines,framesep=5mm,baselinestretch=1.2,fontsize=\footnotesize,linenos,mathescape,escapeinside=??]{py}}{\end{minted}} 

%\newenvironment{bash}{\usemintedstyle{xcode}\VerbatimEnvironment\begin{minted}[frame=lines,framesep=5mm,baselinestretch=1.2,fontsize=\footnotesize,linenos,mathescape]{bash}}{\end{minted}} 

%in order to writ maths outside of comments put them inside ?$$?. Theta is for x^2, write ?$x^2$?.
%\newenvironment{pseudo}{\usemintedstyle{xcode}\VerbatimEnvironment\begin{minted}[frame=lines,framesep=5mm,baselinestretch=1.2,fontsize=\footnotesize,linenos,escapeinside=??,mathescape=true]{python}}{\end{minted}} 

\newenvironment{run}[1]{``{\tt #1}"\xspace}


\newcommand{\dint}{  \displaystyle \int }
%%%%%%%%%%%%%%%%%%%%%%%%%%%%%%%%%%%%%%%%%%
\newcommand{\ie}{{\em i.e.}\xspace}
\newcommand{\eg}{{\em e.g.}\xspace}
\newcommand{\GeV}{{\rm GeV}\xspace}
\newcommand{\TeV}{{\rm TeV}\xspace}
\newcommand{\MeV}{{\rm MeV}\xspace}
\newcommand{\keV}{{\rm keV}\xspace}

\def\nsc{{\tt NSCpp}\xspace}


\newcommand{\CPP}{{\tt C++}\xspace}

\newcommand{\PY}{{\tt python}\xspace}

\newcommand{\JUPY}{{\tt jupyter}\xspace}

\newcommand{\rhs}{RHS\xspace}
\newcommand{\lhs}{LHS\xspace}

\newcommand{\sR}{ s_{\rm R}\xspace}
\newcommand{\HR}{ H_{\rm R}\xspace}
\newcommand{\Hend}{ H_{\rm end}\xspace}
\newcommand{\rhoR}{ \rho_{\rm R}\xspace}
\newcommand{\rhoPhi}{ \rho_{\Phi}\xspace}
\newcommand{\GammaPhi}{ \Gamma_{\Phi}\xspace}


\newcommand{\fR}{ f_{\rm R}\xspace}
\newcommand{\fPhi}{f_{\Phi}\xspace}

\newcommand{\sRi}{ s_{\rm R\,ini}\xspace}
\newcommand{\rhoRi}{ \rho_{\rm R\,ini}\xspace}
\newcommand{\rhoPhii}{ \rho_{\Phi\,{\rm ini}}\xspace}



\newcommand{\geff}{ g_{\rm eff}{}\xspace}
\newcommand{\heff}{ h_{\rm eff}{}\xspace}
\renewcommand{\dh}{ \delta_h{}\xspace}

\newcommand{\Ham}{ \mathcal{H}\xspace}


\newcommand{\Ti}{ T_{\rm ini}{}\xspace}

\newcommand{\ti}{ t_{\rm ini}{}\xspace}

\newcommand{\ui}{ u_{\rm ini}{}\xspace}

\newcommand{\ai}{ a_{\rm ini}{}\xspace}

\newcommand{\Tend}{ T_{\rm end}{}\xspace}

\newcommand{\tend}{ t_{\rm end}{}\xspace}

\newcommand{\uend}{ u_{\rm end}{}\xspace}

\newcommand{\aend}{ a_{\rm end}{}\xspace}




\newcommand{\vev}[1]{\langle #1 \rangle}
\newcommand{\Bvev}[1]{\Bigg\langle #1 \Bigg\rangle}
\newcommand{\bvev}[1]{\Big\langle #1 \Big\rangle}




\newcommand{\lrb}[1]{\left( #1 \right)}
\newcommand{\lrsb}[1]{\left[ #1 \right]}
\newcommand{\lrBigb}[1]{\Big( #1 \Big)}
\newcommand{\lrBigsb}[1]{\Big[ #1 \Big]}
\newcommand{\lrBiggb}[1]{\Bigg( #1 \Bigg)}
\newcommand{\lrBiggsb}[1]{\Bigg[ #1 \Bigg]}

\newcommand{\lrBigcb}[1]{\Big\{ #1 \Big\}}
\newcommand{\lrBiggcb}[1]{\Bigg\{ #1 \Bigg\}}
%%%%%%%%%%%%%%%%%%%%%%%%%%%%%%%%%%%%%%%%%

%%%%%%%%%%%%%%%%%%%%%%%%%%%%%%%%%%%%%%%%%%%%%%%%%%%--Begin_refs--%%%%%%%%%%%%%%%%%%%%%%%%%%%%%%%%%%%%%%%%%%%%%%%%%%%%%%%%%%%%%%%%%%%%%%
\newcounter{NumArgs}

%Define reference to an arbitrary number of equations (\eqs{label_1,label_2....,label_n} will show eqs. ref_1, ref_2, ..., and ref_n)
\newcommand{\eqs}[1]{\setcounter{NumArgs}{0}\foreach\i in{#1}{\stepcounter{NumArgs}}%
	\ifthenelse{\equal{\theNumArgs}{1}}{eq.~(\ref{#1})}%
	{\ifthenelse{\equal{\theNumArgs}{2}}%
		{eqs.~\foreach\i[count=\q]in{#1}{\ifthenelse{\equal{\q}{\theNumArgs}}{and (\ref{\i})}{(\ref{\i})~}}}%
		{eqs.~\foreach\i[count=\q]in{#1}{\ifthenelse{\equal{\q}{\theNumArgs}}{and (\ref{\i})}{(\ref{\i}),~}}}}}


%Define reference to an arbitrary number of equations (\Eqs{label_1,label_2....,label_n} will show Eqs. ref_1, ref_2, ..., and ref_n)
\newcommand{\Eqs}[1]{\setcounter{NumArgs}{0}\foreach\i in{#1}{\stepcounter{NumArgs}}%
	\ifthenelse{\equal{\theNumArgs}{1}}{Eq.~(\ref{#1})}%
	{\ifthenelse{\equal{\theNumArgs}{2}}%
		{Eqs.~\foreach\i[count=\q]in{#1}{\ifthenelse{\equal{\q}{\theNumArgs}}{and (\ref{\i})}{(\ref{\i})~}}}%
		{Eqs.~\foreach\i[count=\q]in{#1}{\ifthenelse{\equal{\q}{\theNumArgs}}{and (\ref{\i})}{(\ref{\i}),~}}}}}


%Define reference to an arbitrary number of labels (\REF{label_1,label_2....,label_n} will show ref_1, ref_2, ..., and ref_n)
\newcommand{\refs}[1]{\setcounter{NumArgs}{0}\foreach\i in{#1}{\stepcounter{NumArgs}}%
	\ifthenelse{\equal{\theNumArgs}{1}}{(\ref{#1})}%
	{\ifthenelse{\equal{\theNumArgs}{2}}%
		{\foreach\i[count=\q]in{#1}{\ifthenelse{\equal{\q}{\theNumArgs}}{and (\ref{\i})}{(\ref{\i})~}}}%
		{\foreach\i[count=\q]in{#1}{\ifthenelse{\equal{\q}{\theNumArgs}}{and (\ref{\i})}{(\ref{\i}),~}}}}}



%Define reference to an arbitrary number of figs (\Figs{label_1,label_2....,label_n} will show ref_1, ref_2, ..., and ref_n)
\newcommand{\Figs}[1]{\setcounter{NumArgs}{0}\foreach\i in{#1}{\stepcounter{NumArgs}}%
	\ifthenelse{\equal{\theNumArgs}{1}}{Fig.~(\ref{#1})}%
	{\ifthenelse{\equal{\theNumArgs}{2}}%
		{Figs.~\foreach\i[count=\q]in{#1}{\ifthenelse{\equal{\q}{\theNumArgs}}{and (\ref{\i})}{(\ref{\i})~}}}%
		{Figs.~\foreach\i[count=\q]in{#1}{\ifthenelse{\equal{\q}{\theNumArgs}}{and (\ref{\i})}{(\ref{\i}),~}}}}}




%Define reference to an arbitrary number of "general reference" (\Gen{message}{label_1,label_2....,label_n} will show message.(ref_1), (ref_2), ..., and (ref_n)
\newcommand{\Gen}[2]{\setcounter{NumArgs}{0}\foreach\i in{#2}{\stepcounter{NumArgs}}%
	\ifthenelse{\equal{\theNumArgs}{1}}{#1.~(\ref{#2})}%
	{\ifthenelse{\equal{\theNumArgs}{2}}%
		{#1.~\foreach\i[count=\q]in{#2}{\ifthenelse{\equal{\q}{\theNumArgs}}{and (\ref{\i})}{(\ref{\i})~}}}%
		{#1.~\foreach\i[count=\q]in{#2}{\ifthenelse{\equal{\q}{\theNumArgs}}{and (\ref{\i})}{(\ref{\i}),~}}}}}


%%%%%%%%%%%%%%%%%%%%%%%%%%%%%%%%%%%%%%%%%%%%%%%%%%%--End_refs--%%%%%%%%%%%%%%%%%%%%%%%%%%%%%%%%%%%%%%%%%%%%%%%%%%%%%%%%%%%%%%%%%%%%%%



%%%%%%%-----------------------Rules-----------------------%%%%%%%
% 1. Put any new macros in macros.tex.

% 2. Section should start with:
	%\section{This is a section}\label{sec:Intro}
	%\setcounter{equation}{0}

% 3. labels for Figs should start with fig:, for equations should start with eq:, for sections with sec:, etc.
%%%%%%%--------------------------------------------------%%%%%%%




\author[ ]{Karamitros Dimitrios}
\affil[ ]{\em Department of Physics and Astronomy, The University of Manchester,}
\affil[ ]{\em Manchester M13 9PL, United Kingdom}

\affil[ ]{}
\affil[ ]{\textit{E-mail: } \href{mailto:dimitrios.karamitros@manchester.ac.uk}{\color{blue}{dimitrios.karamitros@manchester.ac.uk}}}
	
	
\title{\nsc: Non-Standard Cosmologies in {\tt C++}}
\begin{document}

\maketitle


\begin{abstract}
\end{abstract}

{\bf Program summary:}

{\sl 
	Program title: \nsc.
	
	Developer's respository link: \href{https://github.com/dkaramit/NSCpp}{https://github.com/dkaramit/NSCpp}.
	
	Programming language: \CPP and \PY.
	
	Licensing provisions: MIT license.
	
	Nature of problem: 
	
	Solution method: Embedded Runge-Kutta for the numerical integration of the system of differential equations. The user may choose between explicit and Rosenborck methods. There are several of Butcher tableaux already implemented, but the user can implement their own. For the interpolations of the relativistic degrees of freedom, the library uses cubic splines. 
	
	Restrictions: 
}

%\newpage
\tableofcontents
%\newpage




\section{Introduction}\label{sec:intro}
\setcounter{equation}{0}



Several non-standard cosmological scenarios have been studied in the literature (\eg refs.~\cite{Vilenkin:1982wt,Coughlan:1983ci,Ratra:1987rm,McDonald:1989jd,Giudice:2000ex,Gardner:2004in,DEramo:2017gpl,Redmond:2017tja,DEramo:2017ecx,Dalianis:2018afb,Hardy:2018bph,Bernal:2018kcw,Arias:2019uol,Allahverdi:2019jsc,Bernal:2019mhf,Cosme:2020mck,Bernal:2020bfj,Arias:2020qty,Arias:2021rer,Barman:2021ifu,Dienes:2021woi,Banerjee:2022fiw}). Reviews on the implications of such components can be found in refs.~\cite{Tsujikawa:2013fta,Allahverdi:2020bys}.


There is another code, built specifically for the evolution of cosmic relics in matted dominated Universe, {\tt EvoEMD}~\cite{Dutra:2021phm}. \nsc is different because the evolution of the temperature of the plasma is computed by manipulating the equation that describes the evolution of the entropy density, while {\tt EvoEMD} solves
%
\begin{equation}
	\dfrac{d\rhoR}{dt} = -4H\rhoR + C_{\rm EMD} \;,
	\label{eq:EvoMD}
\end{equation}
%
with  $C_{\rm EMD}$ the additional matter contribution which. This equation relies on the approximation $\rhoR \approx 3/4 \, sT$ that does not hold in general. Also, \nsc allows the user to consider different equations of state for the fluid, $\Phi$. 
%
As we will explain later \nsc obeys the same design principles as \mimes; \ie the general usage of both are almost identical, \nsc only needs a \CPP compiler that supports the {\tt c++17} standard and no other external dependences, and there will always be dedicated versions of the underline numerical libraries~\cite{NaBBODES,SimpleSplines} guaranteed to work with the current version of \nsc.



\section{Evolution equations}\label{sec:equations}
\setcounter{equation}{0}

In order to model a non-standard cosmological history, we assume that the Universe at early was dominated by two components; the plasma and a fluid ($\Phi$) with an equation of state
%
\begin{equation}
	p_{\Phi} = (c/3-1) \rhoPhi \;,
	\label{eq:EOS}
\end{equation}  
%
with $c$ a constant. 

The plasma is assumed to be in thermal equilibrium with energy and entropy densities defined as
%
\begin{align}
	\sR(T)&=\dfrac{2\pi^2}{45} \,  \heff(T) \, T^3 \label{eq:s_def}\\
	\rhoR(T)&=\dfrac{\pi^2}{30} \,  \geff(T) \, T^4 \label{eq:rho_def}
\end{align}
%


Since the contribution to the energy budget of the Universe of such fluid is severely constraint at temperatures lower than $T \sim \mathcal{O}(10)~\MeV$, $\Phi$ must lose energy to the plasma. We model this by introducing a constant energy loss rate, $\GammaPhi$. The equations that describe the evolution of both components is
%
\begin{align}
	\dfrac{d\sR(T)}{dt} &= -3 \, H(T)  \, \sR(T) + \dfrac{\GammaPhi}{T} \, \rhoPhi(T) \label{eq:dsdt} \\ 
	\dfrac{d\rhoPhi(T)}{dt} &= - c \, H(T) \, \rhoPhi(T) - \GammaPhi \, \rhoPhi(T) \label{eq:drhoPhidt} \;,
\end{align}
%
with $H$ the Hubble parameter of the Universe, given by
%
\begin{equation}
	H(T) = \sqrt{\dfrac{8}{3\,m_P^2} \ \lrsb{\rhoR(T) + \rhoPhi(T)}}\;,
	\label{eq:H_def}
\end{equation}
%
with $m_P = 1.22 \times 10^{19}~\GeV$.
 

As explained in ref.~\cite{Arias:2020qty}, we can parametrise $\GammaPhi$ in terms of the temperature that $\Phi$ would decay ($\Tend$) if the Universe was radiation dominated, as
%
\begin{equation}
	\GammaPhi \equiv \HR(T=\Tend)\;,
	\label{eq:Tend_def}
\end{equation}
%
where $\HR$ the Hubble parameter for a radiation dominated Universe obtained from \eqs{eq:H_def} with $\rhoPhi=0$.


\subsection{Notation}\label{sec:notation}

In order to solve this system of equations, we change the integration variable to 
%
\begin{equation}
	u=\log\frac{a}{\ai}\;,	
	\label{eq:u_def}
\end{equation}
%
with $a$ the scale factor of the Universe, and $\ai$ its value at some initial time. The initial condition can now be chosen at $\ui=0$, corresponding to $T=\Ti$ and an initial value for $\rhoPhi=\rhoPhii$. 

We can reparametrise the temperature of hte plasma and the energy density of $\Phi$ as 
%
\begin{align}
	T(u) &= \Ti  \, \fR(u) \, e^{-u}  \label{eq:fR_def} \\ 
	\rhoPhi(u) &= \rhoPhii \, \fPhi(u) \label{eq:fPhi_def} \, e^{-cu}\;,
\end{align}
%
with $\fR(u)$ and $\fPhi(u)$ functions of $u$ that obey the initial conditions $\fR(u=0)=\fPhi(u=0)=1$ and the equations
%
\begin{align}
	\dfrac{d\log \fR }{du} &=  1- \lrb{1- \dfrac{\GammaPhi}{H} \, \dfrac{\rhoPhi}{3HTs}  } \dh  \label{eq:dlogfRdu} \\ 
	\dfrac{d\log\fPhi}{du} &= -\dfrac{\GammaPhi}{H} \label{eq:dlogfPhidu} \;,
\end{align}
%
with 
\begin{equation}
	\dh = 1+ \dfrac{1}{3} \dfrac{d\log\heff}{d\log T}\;.
	\label{eq:dh_def}
\end{equation}
%
Notice that for $\GammaPhi=0$ and $\dh=1$, $\fR=\fPhi=1$. 


\nsc solves \eqs{eq:dlogfRdu,eq:dlogfPhidu}, using as user input for $\Tend$, $c$, $\Ti$, and $r=\rhoRi/\rhoPhii$. Then, it automatically transforms them to the temperature of the plasma and the fluid energy density. From these, the user can obtain all other relevant cosmological quantities such as the Hubble parameter or the entropy density of the plasma.



The solution of the system~\refs{eq:dsdt,eq:drhoPhidt} and its implications is studied in the literature (\eg refs.~\cite{Vilenkin:1982wt,Coughlan:1983ci,Ratra:1987rm,McDonald:1989jd,Giudice:2000ex,Gardner:2004in,DEramo:2017gpl,Redmond:2017tja,DEramo:2017ecx,Dalianis:2018afb,Hardy:2018bph,Bernal:2018kcw,Arias:2019uol,Allahverdi:2019jsc,Bernal:2019mhf,Cosme:2020mck,Bernal:2020bfj,Arias:2020qty,Arias:2021rer,Barman:2021ifu,Dienes:2021woi,Banerjee:2022fiw}). The radiation-fluid system changes qualitative behaviour as the Universe expands. \nsc closely follows the notation of ref.~\cite{Arias:2020qty}, which identifies (at most) four points where this behaviour changes; $\EI$, $\DI$, $\EII$, and $\DII$. The point $\EI$ ($\EII$) is defined when $\rhoPhi=\rhoR$ for the first (second) time. The point $\DI$ ($\DII$) is defined as the point at which $\lrsb{ (\Hend/H) \, (\rhoPhi/\rhoR) }$ exceeds (drops below) $40\%$; \ie before $\DI$ and after $\DII$ the energy deposition form the fluid to the plasma has a rate less than $10\%$ of the free dilution rate of $\rhoR$.

There are two qualitatively different cases shown in \Figs{fig:evolution_examples}. 
%
In \Figs{fig:EMD}, we show the evolution of the comoving energy densities of radiation and $\Phi$ as functions of $u$ for $\Tend=10^{-2}~\GeV$, $c=3$, $\Ti=10^7~\GeV$, and $r=10^{-2}$. In \Figs{fig:EKD}, we show the evolution of these quantities for $\Tend=10^{4}~\GeV$, $c=6$, $\Ti=10^7~\GeV$, and $r=10^{8}$. 
%
These two figures correspond to the two quantitatively typical behaviours we expect for $c<4$ and $c>4$. In the former, $\Phi$ starts as a subdominant component with increasing energy contribution, until it decays away. In the latter,  $\Phi$ dominates initially, with a monotonically decreasing contribution to the energy budget of the Universe. In this case, the Universe will become radiation dominated regardless of even if $\GammaPhi$, but if $\rhoPhi$ and $\GammaPhi$ are large enough the energy of the plasma will increse.  Notice that for $c>4$, There is only one point of equality between $\rhoR$ and $\rhoPhi$, which we call $\EI$ as this is the naming conversion in \nsc.
%
\begin{center}
	\begin{figure}[t!]
		\begin{subfigure}{0.5\textwidth}
			\includegraphics[width=1\textwidth]{figs/EMD.pdf}
			\caption{}
			\label{fig:EMD}
		\end{subfigure}
		\begin{subfigure}{0.5\textwidth}
			\includegraphics[width=1\textwidth]{figs/EKD.pdf}
			\caption{}
			\label{fig:EKD}
		\end{subfigure}
		\caption{}
		\label{fig:evolution_examples}
	\end{figure}
\end{center}




\section{\nsc usage}\label{sec:first_steps}
\setcounter{equation}{0}
%
\nsc can be obtained directly form github. The safest choice is to download the stable branch, which will always hold the latest tested version. The preferred method for downloading \nsc is by running:~\footnote{The {\tt git} utility can be installed by following \href{https://github.com/git-guides/install-git}{https://github.com/git-guides/install-git}.}
%
\begin{lstlisting}
	git clone -b stable https://github.com/dkaramit/NSCpp
\end{lstlisting}

\nsc relies on {\tt NaBBODES}~\cite{NaBBODES} and {\tt SimpleSplines}~\cite{SimpleSplines}, which are developed independently. In order to get \nsc with the latest version of these libraries, one needs to run the following commands
%
\begin{lstlisting}
	git clone https://github.com/dkaramit/NSCpp.git
	cd NSCpp
	git submodule init
	git submodule update --remote
\end{lstlisting}
%
These commands download the {\tt master} branch of \nsc, with the latest versions of {\tt NaBBODES} and {\tt SimpleSplines}. All versions that are not labelled as stable must be treated with caution, as they may not work as indented. 

By executing \run{bash configure.sh}, \nsc will be available for usage within a \CPP program. This {\tt bash} script, paths in some files, formats the data for the RDOF in an acceptable format, and creates some directories that will be useful later. In order to call \nsc inside \PY, \run{make lib} must also be ran,  which will compile shared libraries that needed for the \PY interface.

\nsc already comes with several examples in \CPP and \PY, which can be found in the directory {\tt \nsc/UserSpace}. In order to compile the \CPP examples, one just has to run \run{make examples} in the root directory of \nsc or {\tt make} inside the directory of each \CPP program.

There is also an option to compile some test programs using {\tt make check}, which will create executables inside {\tt exec}. These can then be executed, in order to ensure that the code actually works; \eg no segmentation fault occurs. 

\subsection{Assumptions}
%
\nsc is designed to be model agnostic. It can only be used for constant $c$, in cases where there are no plasma energy leaks into the fluid. In cases that these assumptions do not hold, knowledge of the exact nature of $\Phi$ would be needed.  It is also assumed that the plasma does not fall out of thermal equilibrium between the temperatures we integrate (defined later as {\tt Ti} and {\tt TSTOP}), and its energy and entropy densities are always given by \eqs{eq:s_def,eq:rho_def}.

\subsection{Using \nsc}\label{sec:Cpp_usage} 
%
Out-of-the-box, and after \run{bash configure.sh} is ran successfully, all \nsc functionality can be used in a \CPP program by including in the source file the following line:
%
\begin{cpp}
	#include "NSCpp.hpp"
\end{cpp}
%
Notice that if this file is not in the root directory of \nsc, we need to compile it using the flag {\tt -Ipath-to-root}, "path-to-root" the relative (or absolute) path to the root directory of \nsc; \eg if the {\tt .cpp} is in the {\tt NSCpp/UserSpace/Cpp/NSC} directory, this flag should be {\tt -I../../../}. 

 
The class that deals with the system~\refs{eq:dlogfRdu,eq:dlogfPhidu}, is the \cppin{nsc::Evolution<LD,SOLVER,METHOD>}. In order to begin, we declare an instance of this class as
%
\begin{cpp}
    nsc::Evolution<LD,SOLVER,METHOD<LD>> BE;
\end{cpp}
%
Here, \cppin{LD} should be the numeric type to be used; it is recommended to use \cppin{double} or \cppin{long double}. Moreover \cppin{Solver} and \cppin{Method} depend on the type of Runge-Kutta (RK) the user chooses. The available choices are shown in Table~\ref{tab:template-arguments}. 

The evolution of the Universe depends on the evolution of the plasma, which is encoded in the class \cppin{nsc::Cosmo<LD>}. An instance of this class is declared as 
%
\begin{cpp}
	nsc::Cosmo<LD> plasma('path_to_data', minT, maxT);
\end{cpp}
%
Here, \cppin{'path_to_data'} is the (relative of absolute) path to a file with tabulated data  of the relativsit degrees of freedom of the plasma . It is assumed that the file has three columns with: $T$ in $\GeV$, $\heff$, and $\geff$. The temperature given in this file have to increase monotonically, otherwise the the program exits with error core $1$.~\footnote{\nsc already includes the standard model RDOF found in~\cite{Saikawa:2020swg} in the directory {\tt NSCpp/src/data}.} If a path is provided in the {\tt NSCpp/Paths.mk} file before \run{bash configure.sh} is ran, there will be a global (constant) variable called \cppin{cosmo_PATH} which can be used after  {\tt NSCpp.hpp} is included.
%
The arguments \pyin{minT} and \pyin{maxT} dictate the temperatures between which $\geff$ and $\heff$ will be interpolated. For temperatures beyond these, both  $\geff$ and $\heff$  are assumed to be constant. A detailed description of this class can be found in the Appendix of ref.~\cite{Karamitros:2021nxi}.  


The system can be solved by including:
%
\begin{cpp}
	BE.solveNSC(TEND, c, Ti, ratio, TSTOP, umax, &plasma,
				 	 {
					 .initial_step_size=initial_step_size, .minimum_step_size=minimum_step_size,
					 .maximum_step_size=maximum_step_size, .absolute_tolerance=absolute_tolerance, 
					 .relative_tolerance=relative_tolerance, .beta=beta, 
					 .fac_max=fac_max, .fac_min=fac_min, .maximum_No_steps=maximum_No_steps
					 }
					);
\end{cpp}
%
The various parameters are:
%
\begin{enumerate}
	\item {\tt TEND}: the values of the $\Tend$ (in $\GeV$) as introduced in \eqs{eq:dsdt}. 
	
	\item {\tt c}: the parameter that characterises the equation of state of the fluid, defined as in \eqs{eq:EOS}.
	
	\item {\tt Ti}: the initial temperature (in $\GeV$) of integration. Has to be above {\tt TEND}.
	
	\item {\tt ratio}: the value of $\rhoPhi/\rhoR$ at {\tt Ti}. 
	
	\item {\tt TSTOP}: If the temperature drops below this, integration stops. 
	
	\item {\tt umax }: If $u>${\tt umax} the integration stops. It can be used to stop internation if relevant bounds (\eg~\cite{Planck:2018jri}) are violated. It is also a failsafe parameter that ensures that integration stops even in cases where the Universe expands extremely fast.
	
	\item {\tt plasma }: A pointer to an instance of the \cppin{Cosmo} class.
	
	\item {\tt initial\_stepsize} (optional): Size of the initial step the solver takes. 
	
	\item {\tt minimum\_stepsize} (optional): The step-size will be limited to be above number. 
	
	\item {\tt maximum\_stepsize} (optional): The step-size will be limited to be below number. 
	
	\item {\tt absolute\_tolerance} (optional): Absolute tolerance of the RK solver.
	
	\item {\tt relative\_tolerance} (optional): Relative tolerance of the RK solver.
	
	\item {\tt beta} (optional): Aggressiveness of the adaptation process of the solver. Generally, it should be around but less than 1.
	
	\item {\tt fac\_max},  {\tt fac\_min} (optional): The solver will keep step-size to from increasing more than fac\_max and less than fac\_min at each iteration. This makes adaptation of the step-size more stable.
	
	\item {\tt maximum\_No\_steps} (optional): Maximum steps the solver can take. Quits if this number is reached even if the temperature is still larger than {\tt TEND}. 
\end{enumerate}
%
A detailed description of the  effect of the various optional parameters on the operation of the ODE solver can be found in the Appendices of ref.~\cite{Karamitros:2021nxi}. Notice that the optional parameter are passed using aggregation.~\footnote{This means that all arguments inside {\tt \{...\}} define a temporary an instance of a \cppin{struct} that simply holds these arguments.} 


This function returns \cppin{true} (\cppin{false}) if the temperature reached \cppin{TSTOP} for $u<=$\cppin{umax}, which can be used to test whether the integration completed successfully.
%
After this line, the evolution of $T$ and $\rhoPhi$ is stored in \cppin{BE.T} and \cppin{BE.rhoPhi}, with each element corresponding to the value of $u$ the solver takes (accessed as \cppin{BE.u}). The local errors of $T$ and $\rhoPhi$ correspond to the variables \cppin{BE.dT} and \cppin{BE.drhoPhi}. 
%
One can also obtain the points {\tt BE.TE1} ($\TEI$), {\tt BE.TE2} ($\TEII$), {\tt BE.TD1} ($\TDI$), {\tt BE.TD2} ($\TDII$). The corresponding values of $u$ are {\tt BE.uE1}, {\tt BE.uE2}, {\tt BE.uD1}, {\tt BE.uD2}.

\subsection{Using \nsc in \PY}\label{sec:py_usage}
%
The \PY interface modules are in the directory {\tt NSCpp/src/interfacePy}. The usage of the classes is made as similar as possible to the \CPP case. However, one should keep in mind that the various template arguments discussed in the \CPP case have to be chosen at compile-time. That is, for the \PY interface, one needs to choose the numeric type, and RK method to be used when the shared libraries are compiled. This is accomplished by assigning the relevant variables in {\nsc/Definitions.mk} before running \run{make lib}. The various options are the same as in section 4 of ref.~\cite{Karamitros:2021nxi}, and summarised in Table~\ref{tab:compile_time-options}.

The two relevant classes are defined in the module {\tt interfacePy.NSC}, and can be imported in a \PY script as 
%
\begin{py}
	from sys import path as sysPath
	sysPath.append('path_to_src')
	from interfacePy.NSC import NSC 
\end{py}
%
The string {\tt 'path\_to\_src'} must be the relative path to the {\tt NSCpp/src} directory. For example, if the script is located in {\tt NSCpp/UserSpace/Python}, {\tt 'path\_to\_src'} should be {\tt '../../src'}.


We can define an {\tt Evolution} instance as follows 
%
\begin{py}
	BE=Evolution()
\end{py}

An instance of the \pyin{Cosmo} class is declared as 
%
\begin{cpp}
	plasma=Cosmo('path_to_data', minT, maxT)
\end{cpp}
%
The arguments are the same as in \CPP.

The system, them, is solved by running:
%
\begin{cpp}
	BE.solveNSC(TEND, c, Ti, ratio, TSTOP, umax, plasma,
						initial_step_size, minimum_step_size, maximum_step_size, absolute_tolerance, 
						relative_tolerance, beta, fac_max, fac_min, maximum_No_steps)
					  )
\end{cpp}
%
The arguments are identical to the \CPP case, outlined in Table~\ref{tab:NSC-input}, but the instance of \pyin{Cosmo} (\pyin{plasma}) is passed by value. A brief description of this function can also be found by running \pyin{?BE.solveNSC} after loading the module. This function returns \cppin{true} (\cppin{false}) if the temperature reached \cppin{TSTOP} for $u<=$\cppin{umax}, and the time it took to execute it in seconds.
%
In contrast to the \CPP implementation, this only gives us access to the points where the behaviour changes; the corresponding variables are {\tt BE.TE1}, {\tt BE.TE2}, {\tt BE.TD1}, {\tt BE.TD2}, {\tt BE.uE1}, {\tt BE.uE2}, {\tt BE.uD1}, and {\tt BE.uD2}. In order to get the evolution of $T$ and $\rhoPhi$, we need to run 
%
\begin{py}
	BE.getPoints()
\end{py}
%
This will fill the \pyin{numpy}~\cite{harris2020array} arrays with the integration variable and the solutions; \pyin{BE.u}, \pyin{BE.T}, and \pyin{BE.rhoPhi}. The following line:
%
\begin{py}
	BE.getErrors()
\end{py}
%
gives access to the local errors \pyin{BE.dT} and \pyin{BE.drhoPhi}.

It is essential to manually delete both \pyin{BE} and \pyin{plasma} after they are no longer needed. The reason is that there are underlying {\tt C}-pointers that manage the \PY interface for each instance of the classes. If the instances are reassigned without deletion, this pointers can no longer be accessed. This is a memory-leak, which can only be fixed by exiting the script. 
%
That is, one must simply run 
%
\begin{py}
	del BE, plasma
\end{py}  
%
after {\tt BE} served its purpose.

\section{Acknowledgements}
%
The author acknowledges support by the Lancaster–Manchester–Sheffield Consortium for Fundamental Physics, under STFC research grant ST/T001038/1.


\section{Summary}


%%%%%%%%%%%%%%%%%%%%%%%%%%%%%%%%%%%%%%%%
% APPENDICES
\pagebreak
%%%%%%%%%%%%%%%%%%%%%%%%%%%%%%%%%%%%%%%%
\setcounter{section}{0}
\section*{Appendix}
\appendix

\renewcommand{\theequation}{\Alph{section}.\arabic{equation}}
\setcounter{equation}{0}  % reset counter
%%%%%%%%%%%%%%%%%%%%%%%
In the appendix, we show a detailed description of the \cppin{Evolution} class in both \CPP and \PY. A detailed description of the \cppin{Cosmo} class as well as a basic introduction of the RK methods used can be found in ref.~\cite{Karamitros:2021nxi}.


\section{\cppin{Evolution} class in \CPP}\label{app:cpp}
\setcounter{equation}{0}
The \cppin{Evolution} class uses the default constructors, so instances are declared trivially as shown in Section~\ref{sec:first_steps}. It has only one public member function, \cppin{solveNSC}, with signature
%
\begin{cpp}
	template<class LD, const int Solver, class Method>
	bool Evolution<LD,Solver,Method>::solveNSC(const LD &TEND, const LD &c, const LD &Ti, 
							 const LD &ratio, const LD &TSTOP, const LD &umax, Cosmo<LD> *plasma, 
							 const solverArgs<LD> &args={})
\end{cpp}
%
All the template and required input arguments are summarised in Table~\ref{tab:solveNSC-input}. This function returns \cppin{false} if the integration stops above \cppin{TSTOP} and \cppin{true} otherwise, as a way of providing a simple consistency check. 
%
The optional argument, \cppin{args}, is an instance of a \cppin{struct} defined as 
%  
\begin{cpp}
 	namespace nsc{
	 	template<class LD>
		struct solverArgs{
			LD initial_step_size=1e-2, minimum_step_size=1e-8, maximum_step_size=1e-2; 
			LD absolute_tolerance=1e-8, relative_tolerance=1e-8;
			LD beta=0.9, fac_max=1.2, fac_min=0.8;
			unsigned int maximum_No_steps=10000000;
		};
	}
\end{cpp}
%
This helps us pass to \cppin{solveNSC} only the arguments we wish to change. For example, we can choose {\tt absolute\_tolerance=1e-11} and {\tt beta=0.8} by adding {\tt \{.absolute\_tolerance=1e-11, .beta=0.8\}} in the place of {\tt args}. This is called aggregation in \CPP, and basically allows us to use named arguments. The caveat is that we need to put the arguments in the order thy appear in the definition of \cppin{solverArgs}.


The class \cppin{Evolution} has several member variables that store the results of the solution of the system~\refs{eq:dlogfRdu,eq:dlogfPhidu}:
%
\begin{enumerate}
	\item \cppin{std::vector<LD> u}: integration steps of $u$.
	\item \cppin{std::vector<LD> T}: values of $T$ (in $\GeV$) that corresponds to every step in \cppin{u}.
	\item \cppin{std::vector<LD> rhoPhi}: values of $\rhoPhi$ (in $\GeV^4$) that corresponds to every step in \cppin{u}.
	\item \cppin{std::vector<LD> dT}: local integration error of $T$ (in $\GeV$) at every \cppin{u}.
	\item \cppin{std::vector<LD> drhoPhi}: local integration error of $\rhoPhi$ (in $\GeV^4$) at every \cppin{u}.
	%%%%%%%%%%%%%%%%%%%%%%%%%%%%%%%%%%%%%%%%%%%%%%%%%%%%%%%%%%%%%%%%%%%%%%%%%%%%%%%%%%%%%%%%%%%%%%%%%%%%%%%%%%%%
	\item \cppin{LD TE1}: value of $\TEI$ (in $\GeV$). It is initialised at 0. If $\rhoPhi$ never reaches $\rhoR$, This variable will be equal to $\Ti$.
	\item \cppin{LD TE2}: value of $\TEII$ (in $\GeV$). It is initialised at 0. If $\rhoPhi$ never reaches $\rhoR$ or $c>4$, this variable will be equal to $\Ti$. 
	\item \cppin{LD TD1}: value of $\TDI$ (in $\GeV$). It is initialised at 0. If the energy injection rate never becomes more than $10\%$ of the free dilution rate of $\rhoR$, this variable obtains the value of $\Ti$.
	\item \cppin{LD TD2}: value of $\TDII$ (in $\GeV$).It is initialised at 0. If the energy injection rate never becomes more than $10\%$ of the free dilution rate of $\rhoR$, this variable obtains the value of $\Ti$.
	%%%%%%%%%%%%%%%%%%%%%%%%%%%%%%%%%%%%%%%%%%%%%%%%%%%%%%%%%%%%%%%%%%%%%%%%%%%%%%%%%%%%%%%%%%%%%%%%%%%%%%%%%%%%
	\item \cppin{LD uE1}: value of $\uEI$. It is initialised at 0. If $\rhoPhi$ never reaches $\rhoR$, This variable will be equal to $0$.
	\item \cppin{LD uE2}: value of $\uEII$. It is initialised at 0. If $\rhoPhi$ never reaches $\rhoR$  or $c>4$, this variable will be equal to $0$.
	\item \cppin{LD uD1}: value of $\uDI$. It is initialised at 0. If the energy injection rate never becomes more than $10\%$ of the free dilution rate of $\rhoR$, this variable obtains the value of $0$.
	\item \cppin{LD uD2}: value of $\uDII$.It is initialised at 0. If the energy injection rate never becomes more than $10\%$ of the free dilution rate of $\rhoR$, this variable obtains the value of $0$.	
\end{enumerate} 



\section{\cppin{Evolution} class in \PY }\label{app:py}
\setcounter{equation}{0}
The \pyin{Evolution} class is very similar to the \CPP case. It uses the default constructor, with instances declared as shown in Section~\ref{sec:first_steps}. The pyin{solveNSC}, has signature
%
\begin{cpp}
	solveNSC(TEND, c, Ti, ratio, TSTOP, umax, plasma, 
		initial_step_size=1e-2, minimum_step_size=1e-8, maximum_step_size=1e-2, 
		absolute_tolerance=1e-8, relative_tolerance=1e-8, beta=0.9, fac_max=1.2, fac_min=0.8,
		maximum_No_steps=10000000)
\end{cpp}
%
All the input arguments are summarised in Table~\ref{tab:solveNSC-input}. The difference between this and the \CPP version is that the {\tt plasma} argument is an instance of the \PY \cppin{Cosmo} class. 
%
This function returns a tuple. The first element of the returned tuple is \pyin{False} if the integration stops above \cppin{TSTOP} and \pyin{True} otherwise, while the second is the time it took to solve the system (in seconds).

The member variables are the same as in \CPP, with {\tt numpy} arrays used in place of the \CPP vectors. However, solving the system using \pyin{solveNSC} does not fill the arrays with the integration steps. The member function
%
\begin{py}
	getPoints()
\end{py}
%
Fills the arrays {\tt u}, {\tt T}, and {\tt rhoPhi} with the integration steps.


In order to fill the arrays with the local errors we make use of the member function
%
\begin{py}
	getErrors()
\end{py}


\section{Quick guide to the user input}\label{app:usr_input}
\setcounter{equation}{0}
In this section, we  present tables with all the inputs the various functions need.
%
\begin{table}[h!]
	\centering
	\begin{tabular}{l l}
		
		\hline\\[-0.4cm]
		\multicolumn{2}{c}{\bf User input for interpolating RDOF.}  \\
		\hline\\[-0.4cm]
		
		{\tt path} & \multirow{1}{12cm}{Path to a file with $T$ ($\GeV$), $\heff$, and $\geff$. The temperatures have to monotonically increase.}\\\\
		\hline\\[-0.4cm]
		
		{\tt minT} & \multirow{1}{12cm}{The RDOF will be interpolated starting from the closest temperature that satisfies $T_{\rm min}\leq${\tt minT}. Below this, both $\heff$ and $\geff$ will be assumed to be their value at $T_{\rm min}$.}\\\\\\
		\hline\\[-0.4cm]
		
		{\tt maxT} & \multirow{1}{12cm}{The RDOF will be interpolated up to the closest temperature that satisfies $T_{\rm max}\geq${\tt maxT}. Above this, both $\heff$ and $\geff$ will be assumed to be their value at $T_{\rm max}$.}\\\\\\
		\hline\\[-0.4cm]
	\end{tabular}
	\caption{Table of the arguments of the \cppin{nsc::Cosmo<LD>} class constructor.}
	\label{tab:Cosmo-input}
\end{table}
%
\begin{table}[p]
	\centering
	\begin{tabular}{l l}
		\hline\\[-0.4cm]
		\multicolumn{2}{c}{\bf User input for solving \eqs{eq:dlogfRdu,eq:dlogfPhidu}.}  \\
		\hline\\[-0.4cm]
		{\tt TEND} & The value of $\Tend$ in $\GeV$.\\
		\hline\\[-0.4cm]
		
		{\tt c} & Value of $c$ as defined through~\eqs{eq:EOS}.\\
		\hline\\[-0.4cm]
		
		{\tt Ti} & The value of $\Ti$ in $\GeV$.\\
		\hline\\[-0.4cm]

		{\tt ratio} & The value of $\rhoPhi/\rhoR$ at $\Ti$.\\
		\hline\\[-0.4cm]

		{\tt plasma} & A pointer to an instance of the {\tt Cosmo} class.\\
		\hline\\[-0.4cm]

		{\tt TSTOP} & Once $T<${\tt TSTOP}, integration stops.\\
		\hline\\[-0.4cm]
		
		{\tt umax } & If $u>${\tt umax} the integration stops.\\
		\hline\\[-0.4cm]
		
		{\tt initial\_stepsize} &  Initial step-size of the solver. Default value: $10^{-2}$.\\ 
		\hline\\[-0.4cm]

		{\tt minimum\_stepsize} & Lower limit of the step-size. Default value:  $10^{-8}$.\\
		\hline\\[-0.4cm]

		{\tt maximum\_stepsize} & Upper limit of the step-size. Default value:  $10^{-2}$.\\
		\hline\\[-0.4cm]

		{\tt absolute\_tolerance} & \multirow{1}{12cm}{Absolute tolerance of the RK solver.  Default value:  $10^{-8}$.}\\\\
		\hline\\[-0.4cm]

		{\tt relative\_tolerance} & \multirow{1}{12cm}{Relative tolerance of the RK solver.  Default value:  $10^{-8}$.}\\\\
		\hline\\[-0.4cm]
		
		{\tt beta} & \multirow{1}{12cm}{Aggressiveness of the adaptation strategy.  Default value:  $0.9$.}\\\\
		\hline\\[-0.4cm]

		{\tt fac\_max}, {\tt fac\_min} &\multirow{1}{12cm}{The step-size does not change more than {\tt fac\_max} and less than {\tt fac\_min} within a trial step . Default values: $1.2$ and $0.8$, respectively.} \\ \\ \\ 
		\hline\\[-0.4cm]
		
		{\tt maximum\_No\_steps} & \multirow{1}{12cm}{If integration needs more than {\tt maximum\_No\_steps} integration stops. Default value: $10^7$.}\\\\
		\hline\\[-0.4cm]
	\end{tabular}
	\caption{Table of the arguments of the \cppin{nsc::Evolution<LD,Solver,Method>::solveNSC} method.}
	\label{tab:solveNSC-input}
\end{table}



\begin{table}[p]
	\centering
	\begin{tabular}{l l}
		\multicolumn{2}{c}{\bf Template arguments.}  \\
		\hline\\[-0.4cm]
		
		{\tt LD}& \multirow{1}{12cm}{This template argument is the numeric type that \nsc will use. The preferred choice is \cppin{long double}. However, in many cases \cppin{double} can be used. Notice that the instance of the \cppin{Cosmo} class must use identical numeric type to the one that used by the instance of \cppin{Evolution}, in order to be able to run the \cppin{solveNSC} method.}\\\\\\\\\\		
		\hline\\[-0.4cm]
		
		{\tt Solver}& \multirow{1}{12cm}{This is the second template argument of the \cppin{nsc::Evolution<LD,Solver,Method>} class. The available choices are {\tt Solver}=$1$ for Rosenbrock method, and {\tt Solver}=$2$ for explicit RK method.}\\\\\\\\
		\hline\\[-0.4cm]
		
		{\tt Method}& \multirow{1}{12cm}{The third template argument of the \cppin{Evolution} class. Its value depends on the choice of \cppin{Solver}; For {\tt Solver}=$1$, {\tt Method} can be \cppin{RODASPR2<LD>} (fourth order)~\cite{RangAngermann2005}, \cppin{ROS34PW2<LD>} (third order)~\cite{RANG2015128}, \cppin{GRK4A<LD>} or \cppin{GRK4T<LD>} (fourth order)~\cite{Rentrop1979}. For {\tt Solver}=$2$, there are a few options, but the only choice that works in most cases is \cppin{DormandPrince<LD>} (fourth order)~\cite{DORMAND198019}. The {\tt Method} classes need a template argument, \cppin{LD}, which must be the same as the first template argument of the \cppin{nsc::Evolution<LD,Solver,Method>} class. If one defines their own Butcher table, they would have to follow their definitions and assumptions.}\\\\\\\\\\\\\\\\\\\\
		\hline
	\end{tabular}
	\caption{Template arguments of the various \nsc classes.}
	\label{tab:template-arguments}
\end{table}




\begin{table}[p]
	\centering
	\begin{tabular}{l l}
		\multicolumn{2}{c}{\bf User compile-time options. Variables in the various {\tt Definitions.mk} files.}  \\
		\hline\\[-0.4cm]

		{\tt rootDir}& \multirow{1}{12cm}{The relative path of root directory of \nsc. Relevant only when compiling using {\tt make}. Available in all {\tt Definitions.mk}.}\\\\		
		\hline\\[-0.4cm]
		
		{\tt LONG}& \multirow{1}{12cm}{{\tt long} for \cppin{long double} or empty for \cppin{double}. This is defines a macro in the source files of the various \CPP examples. Available in {\tt Definitions.mk} inside the various subdirectories of {\tt \nsc/UserSpace/Cpp}.}\\\\\\\\		
		\hline\\[-0.4cm]

		{\tt LONGpy}& \multirow{1}{12cm}{{\tt long} or empty. Same as {\tt LONG}, applies to the \PY modules. Available in {\tt \nsc/Definitions.mk}.}\\\\		
		\hline\\[-0.4cm]

		{\tt SOLVER}& \multirow{1}{12cm}{In order to use a Rosenbrock method {\tt SOLVER}=$1$. For explicit RK method, {\tt SOLVER}=$2$. This defines a macro that is passes as the second template argument of \cppin{nsc::Evolution<LD,Solver,Method>}.  The corresponding variable in {\tt \nsc/Definitions.mk} applies to the \PY modules. The variable in {\tt \nsc/UserSpace/Cpp/NSC/Definitions.mk} applies to the example in the same directory.}\\\\\\\\\\\\\\		
		\hline\\[-0.4cm]

		{\tt METHOD}& \multirow{1}{12cm}{Depending on the solver, this variable should name one of its available methods. For {\tt SOLVER}=$1$, {\tt METHOD}={\tt RODASPR2}(fourth order) or {\tt ROS34PW2}(third order). For {\tt SOLVER}=$2$, {\tt METHOD}={\tt DormandPrince }(seventh order). There is a macro ({\tt METHOD}) used by the shared library {\tt \nsc/lib/libNSC.so}. The corresponding variable in {\tt \nsc/Definitions.mk} applies to the \PY modules. The variable in {\tt \nsc/UserSpace/Cpp/NSC/Definitions.mk} applies to the example in that directory.}\\\\\\\\\\\\\\\\
		 		
		\hline\\[-0.4cm]
		
		\multicolumn{2}{c}{\bf Compiler options}  \\
		\hline\\[-0.4cm]
		
		{\tt CC} &  \multirow{1}{12cm}{The preferred \CPP compiler ({\tt g++} by default). Corresponding variable in all {\tt Definitions.mk} files.} \\\\
		\hline\\[-0.4cm]
		
		{\tt OPT} &  \multirow{1}{12cm}{Available options are {\tt OPT}={\tt O1}, {\tt O2}, {\tt O3} (be default). This variable defines the optimization level of the compiler. The variable can be changed in all {\tt Definitions.mk} files. In the root directory of \nsc, the optimization level applies to the python modules (\ie the shared libraries), while in the subdirectories of {\tt \nsc/UserSpace/Cpp} it only applies to example inside them.}   \\\\\\\\\\\\
		\hline\\[-0.4cm]

	\end{tabular}
	\caption{User compile-time input and options. These are available in the various {\tt Definitions.mk} files, which are used when compiling using {\tt make}.}
	\label{tab:compile_time-options}
\end{table}



%%%%%%%%%%%%%%%%%%%%%%%%%%%%%%%
\pagebreak
\bibliography{refs}{}
\bibliographystyle{JHEP}                        

\end{document}
